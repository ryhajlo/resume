% LaTeX resume using res.cls
\documentclass[line,margin]{res} 
%\usepackage{helvetica} % uses helvetica postscript font (download helvetica.sty)
%\usepackage{newcent}   % uses new century schoolbook postscript font 
\setlength{\textheight}{10in}
\setlength{\textwidth}{5.5in}
\setlength{\oddsidemargin}{-0.3in}
\begin{document}

\name{Nicholas Ryhajlo}
% \address used twice to have two lines of address
\address{2226 Pacific Ave Apt E, Costa Mesa, CA 92627}
\address{(949) 293-4039 -  nick.ryhajlo@gmail.com}

 
\begin{resume}
 
\section{Education} {\sl Master of Science,} Computer Science \\
                % \sl will be bold italic in New Century Schoolbook (or
	        % any postscript font) and just slanted in
		% Computer Modern (default) font
                Montana State University, Bozeman, MT, July 2013 \\
                Thesis Title: \textit{Fuzzy Bayesian Networks for Prognostics and Health Management}

				{\sl Bachelor of Arts,} Computer Science \& {\sl Bachelor of Arts,} Mathematics \\
                % \sl will be bold italic in New Century Schoolbook (or
	        % any postscript font) and just slanted in
		% Computer Modern (default) font
                Carroll College, Helena, MT, 
                May 2011
                
\section{Experience}

				{\sl Senior Software Engineer} \hfill Winter 2014 - Present \\
				 Tyvak Nano-Satellite Systems, Irvine CA
				\begin{itemize}  \itemsep -2pt % reduce space between items
					\item Co-architect and implement first and second generation flight software platform used on multiple successful missions.
					\item Designed and implemented C++ wrapper for Maltab/Simulink autocoded ADCS system.
					\item Implemented and physically buillt next generation GNC hardware in the loop system using Speedgoat Real-time target computer and Matlab/Simulink.
					\item Responsable software engineer for ADCS system.
					\item Responsable for on time delivery and on orbit performance of flight software system for multiple successful missions: Nanoace, Tyvak 53b, Tyvak 61c.
					\item Interface directly with customers to determine payload and mission requirements then design, implement, test, launch, and operate those missions.
					\item Heavily involved in other successful missions: Tyvak 0129, RainCube, CICERO Demo, CICERO 4, CICERO 5, CICERO 6, PROPCUBE 1, PROPCUBE 2, PROPCUBE 3.
					\item Experience with all aspects of flight software, focus on camera payload interactions, computer vision, guidance navigation and control, and embedded systems.
					\item Helped design in house ground station management and TT\&C software.
					\item Operations support, especially during LEOP.
					\item Mentor and guide junior software engineers.
					\item Assembly, Integration and Test experience - assembled one spacecraft that is now on orbit, developed cleaning and cleanliness procedures used at Tyvak.
				\end{itemize}

				{\sl Software Engineer} \hfill Summer 2013 - Winter 2014 \\
                  406 Aerospace, Bozeman MT
                 \begin{itemize}  \itemsep -2pt % reduce space between items
                 \item Develop generic hot swappable Linux camera driver interface using V4L2 for Aptina MT9T111, Aptina MT9M021 visible and FLIR Tau 2 640 infrared cameras. 
                 \item Develop C\# (Mono) solution to support Computer Vision applications in an embedded environment for CPOD Mission.
                 \item Design flexible architecture for space based computer vision system with messaging and onboard database integration.
                 \item Develop visible and infrared camera calibration and focusing procedures as well as performance tests.
                 \item Experience with TI DM3730 Arm Cortex A8.
                 
                \end{itemize}


                 {\sl Graduate Research Assistant} \hfill Fall 2011 - Spring 2013 \\
                Space Science and Engineering Laboratory,
                Montana State University, Bozeman MT
                 \begin{itemize}  \itemsep -2pt % reduce space between items
                 \item Lead Embedded Systems, Flight Software and Ground Station developer for the Dual-CubeSat FIREBIRD Mission.
                 \item Designed and implemented a $\mu$C/OS-II based flight software system on a PIC24F microprocessor.
	      		 \item Designed and implemented a FAT-like filesystem for high-speed data storage to NAND Flash for science data.
	      		 \item Provided technical support and design guidance for follow-on FIREBIRD II mission.
                 
                \end{itemize}
 
                %{\sl Montana Space Grant Consortium Summer Intern} \hfill            Summer 2010, 2011 \\
                %Space Science and Engineering Laboratory,
                %Montana State University, Bozeman, MT
                % \begin{itemize}  \itemsep -2pt %reduce space between items
                % \item  Developed hardware drivers for the PIC24F
                % \end{itemize} 
                 
                %{\sl Lewis and Clark County I.T. Intern} \hfill        Summer 2008 - Spring 2011 \\
                %Lewis and Clark County IT{\&}S,
				%Helena, MT
                %  \begin{itemize}
                %   \item Installed and maintained computers throughout Lewis and Clark County.
                %   \item Maintained an inventory of surplussed hardware, and prepped the hardware for resale according to Criminal Justice Information Network standards.
                %   \end{itemize} 

\clearpage

\section{Honors and Awards}           
			2013 IEEE AUTOTESTCON Best Student Paper \\  
			2011 Mathematical Contest in Modeling - Honorable Mention \\
            2010 Mathematical Contest in Modeling - Meritorious Winner \\
            2010 University Physics Competition - Bronze Medal \\
            %2010 William Lowell Putnam Mathematical Contest - Participant \\
			2009 Mathematical Contest in Modeling - Meritorious Winner \\
            %2009 William Lowell Putnam Mathematical Contest - Participant \\
            2006 Eagle Scout
            
\section{Knowledge and Skills} {\sl Languages, Software \& Embedded Systems:}\\
				C++, C, Python, MATLAB, Simulink, Bash, C\#, \LaTeX, FreeRTOS, Linux, Arduino, Raspberry PI, PIC24, ARM Cortex, MSP430, I2C, UART, SPI, git, SVN, Mercurial
                
                {\sl Topics \& Techniques:}\\Data Structures, Software Engineering, Embedded Software Development, Computer Network Design and Routing, Prognostics and Health Management
                
                {\sl Machine Learning Concepts:}\\Artificial Neural Networks, Bayesian Networks, Evolutionary Computation, Swarm Intelligence, Reinforcement Learning, Fuzzy Logic
			
\section{Publications} 
{\sl First Multipoint In Situ Observations of Electron Microbursts: Initial Results From the NSF FIREBIRD-II Mission}\\
Journal of Geophysical Research: Space Physics

{\sl Flight System Technologies Enabling the Twin-CubeSat FIREBIRD-II Scientific Mission}\\
AIAA/USU Conference on Small Satellites. Volume: SSC15-V-6 

{\sl Diagnostic Bayesian Networks with Fuzzy Evidence}\\
AUTOTESTCON, 2013 IEEE

{\sl Fuzzy Bayesian Networks for Prognostics and Health Management}\\	

\section{On-orbit Spacecraft}
{\sl Nanoace} \hfill Mission Responsable Software Engineer\\
{\sl Tyvak 53b} \hfill Mission Responsable Software Engineer\\
{\sl Tyvak 61c} \hfill Mission Responsable Software Engineer\\
{\sl Tyvak 0129} \hfill System architecture development + operations\\
{\sl RainCube} \hfill Payload camera and system architecture development + LEOP\\
{\sl CICERO Demo} \hfill System architecture development + operations\\
{\sl CICERO 4} \hfill System architecture development + operations\\
{\sl CICERO 5} \hfill System architecture development + operations\\
{\sl CICERO 6} \hfill System architecture development + operations\\
{\sl PropCube 1} \hfill Operations Support\\
{\sl PropCube 2} \hfill System architecture updates\\
{\sl PropCube 3} \hfill System architecture updates\\
{\sl FIREBIRD FLT1} \hfill Responsable Software Engineer + software architecture\\
{\sl FIREBIRD FLT2} \hfill Responsable Software Engineer + software architecture\\
{\sl FIREBIRD II FLT3} \hfill Software architecture\\
{\sl FIREBIRD II FLT4} \hfill Software architecture\\
{\sl EDSN} \hfill Payload embedded software\\
			
\end{resume}
\end{document}
